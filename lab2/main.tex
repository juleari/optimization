\documentclass[12pt]{article}

\usepackage{ucs}
\usepackage{amsmath}                    % gather* для вывода формул по центру страницы
\usepackage[utf8x]{inputenc}        % Включаем поддержку UTF8
\usepackage[russian]{babel}         % Включаем пакет для поддержки русского языка

\usepackage[
    left=2cm,           % Поле левое : 200 мм
    right=2cm,          % Поле правое : 200 мм
    top=2cm,            % Поле верхнее: 200 мм
    bottom=2cm,         % Поле нижнее : 200 мм
    bindingoffset=0cm]{geometry}

\setlength{\parindent}{1.25cm}          % Абзацный отступ: 1,25 см
\usepackage{indentfirst}                % 1-й абзац имеет отступ

\usepackage[pdftex]{graphicx, color}
\usepackage{color}
\usepackage{tikz}
\usepackage{url}            % использование URL в библиографии
\usepackage{listings}           % использование листингов кода
\usepackage[nooneline]{caption}
\captionsetup[table]{justification=raggedleft}
\captionsetup[figure]{justification=centering,labelsep=endash}
%\usepackage{array}

\usepackage[nodisplayskipstretch]{setspace}
\setstretch{1.5}

\usepackage{caption}
\usepackage{graphicx}
\usepackage{subcaption}
\usepackage{cases}

\renewcommand{\baselinestretch}{1.5}

% вставка листингов с кодом
\lstset{inputencoding=utf8x,
        extendedchars=false,
        keepspaces=true,
        language=Python}

\renewcommand{\lstlistingname}{Листинг}

\setcounter{tocdepth}{4}    % chapter, section, subsection, subsubsection и paragraph
\setcounter{secnumdepth}{4}

\parindent=1,25cm               % красная строка = 1 см
\usepackage{enumitem}
\setlist[enumerate,1]{leftmargin=2.25cm}
\setlist[itemize]{leftmargin=2.25cm}
\graphicspath{{pics/}}
\DeclareGraphicsExtensions{{.jpg}}


\begin{document}

    \thispagestyle{empty}
    \newpage{
        \centering
            \textbf{
                МОСКОВСКИЙ ГОСУДАРСТВЕННЫЙ ТЕХНИЧЕСКИЙ УНИВЕРСИТЕТ ИМЕНИ Н. Э. БАУМАНА \\
                Факультет информатики и систем управления \\
                Кафедра теоретической информатики и компьютерных технологий}

            \vfill

            \vfill

            {\large\bf Численные методы поиска безусловного экстремума} \\
            \vfill

            \hfill\parbox{7cm} {
                Исполнитель: Ю.А. Волкова \\
                Группа: ИУ9-111
            }

            \vspace{\fill}

            25 Декабря, \number\year
            \clearpage
    }
        \newpage

        \section{Метод модифицированных функций Лагранжа}
    \subsection{Постановка задачи}
        Даны дважды непрерывно дифференцируемые целевая функция $f(x)=f(x_1 \ldots x_n)$ и функции ограничений $g_j(x) = 0, j=1 \ldots m$ и $g_j(x) \leq 0, j=m+1 \ldots p$, определяющие множество допустимых решений $X$.

        Требуется найти локальный минимум целевой функции на множестве $X$, т.е. такую точку $x^e \in X$, что

        \begin{gather}
        f(x^e) = \min_{x \in X}f(x) \nonumber
        \end{gather}
        \begin{numcases}{}
                g_j(x) = 0,    j = 1 \ldots m; m < n \nonumber \\
                g_j(x) \leq 0, j = m + 1 \ldots p \nonumber
        \end{numcases} \nonumber

        Где $f(x), g(x)$:
        \begin{gather}
            f(x) = x_1^2 + x_2^2 \rightarrow min \nonumber \\
            g_1(x) = -x_1 + 1 \leq 0 \nonumber \\
            g_2(x) = x_1 + x_2 - 2 \leq 0 \nonumber 
        \end{gather}

    \subsection{Код программы}
        \lstinputlisting[language=Python]{3.2/lab.py}

    \subsection{Результат работы программы}
        Для поиска безусловного минимума используется программа, написанная для лабораторной работы 2.2.

        В результате работы программы получена точка [1.0, 0.0].
        Значение функции в этой точке: 1.0.
        \newpage
        \section{Метод модифицированных функций Лагранжа}
    \subsection{Постановка задачи}
        Даны дважды непрерывно дифференцируемые целевая функция $f(x)=f(x_1 \ldots x_n)$ и функции ограничений $g_j(x) = 0, j=1 \ldots m$ и $g_j(x) \leq 0, j=m+1 \ldots p$, определяющие множество допустимых решений $X$.

        Требуется найти локальный минимум целевой функции на множестве $X$, т.е. такую точку $x^e \in X$, что

        \begin{gather}
        f(x^e) = \min_{x \in X}f(x) \nonumber
        \end{gather}
        \begin{numcases}{}
                g_j(x) = 0,    j = 1 \ldots m; m < n \nonumber \\
                g_j(x) \leq 0, j = m + 1 \ldots p \nonumber
        \end{numcases} \nonumber

        Где $f(x), g(x)$:
        \begin{gather}
            f(x) = x_1^2 + x_2^2 \rightarrow min \nonumber \\
            g_1(x) = -x_1 + 1 \leq 0 \nonumber \\
            g_2(x) = x_1 + x_2 - 2 \leq 0 \nonumber 
        \end{gather}

    \subsection{Код программы}
        \lstinputlisting[language=Python]{3.2/lab.py}

    \subsection{Результат работы программы}
        Для поиска безусловного минимума используется программа, написанная для лабораторной работы 2.2.

        В результате работы программы получена точка [1.0, 0.0].
        Значение функции в этой точке: 1.0.
        \newpage
        \section{Постановка задачи}
    Дана модифицированная функция Розенброка:

    \begin{gather}
        f(x_1, x_2) = a(x_1^2 - x_2)^2 + b(1 - x_1)^2 - cx_2^3x_1 \nonumber
    \end{gather}

    Множество допустимых решений:
    \begin{gather}
        [x_i^{min}, x_i^{max}] = [-2, 2] \nonumber
    \end{gather}

    Исследовать характер решений в зависимости от параметров $a, b, c$ при значениях:
    $a = (50, 100, 250); b = (5, 8, 10); c = (0.5, 1.0, 2.0)$.

    Требуется найти глобальный экстремум $X^* \in {X^e}$ -- (${X^e}$ -- множество локальных экстремумов) на множестве $D$.
\section{Код программы}
    \lstinputlisting[language=Python]{lab.py}

\section{Результат работы программы}
    Результат работы программы представлен в таблице в зависимости от параметров:

    \begin{tabular}{|l|l|l|l|l|}\hline
        a & b & c & $[x_1, x_2]$ & f \\ \hline
        50 & 5 & 0.5 & [1.4068860283340956, 1.9988157589133708] & -4.7707842184979095 \\ \hline
        100 & 8 & 1.0 & [1.4049365385116972, 1.9981178246862883] & -9.837092457332599 \\ \hline
        250 & 10 & 2.0 & [1.4220618355043535, 1.999313468375218] & -20.81657024214862 \\ \hline
    \end{tabular}
        
\end{document}