\section{Методы нулевого порядка. Метод деформируемых симплексов (Метод Недлера-Мида)}
    \subsection{Постановка задачи}
        Найти безусловный минимум функции $f(x)$ многих переменных, т.е. найти такую точку $x^e \in R^n$, что 
        \begin{gather}
        f(x^e) = \min_{x \in R^n}f(x) \nonumber \\
        f(x) \in C^0(X) \nonumber
        \end{gather}

        Где $f(x)$ задана уравнением:
        \begin{gather}
        f(x) = 4(x_1 - 5)^2 + 2(x_2 - 6)^4 \nonumber
        \end{gather}

    \subsection{Код программы}
        \lstinputlisting[language=Python]{2.1/lab.py}
    \subsection{Результат работы программы}
        При значениях коэффициентов $\alpha = 1, \beta = 0.5, \gamma = 2$ и при $\epsilon = 0.01$, получили точку $x = [4.997193505583281, 5.93581556216305]$.
        Значение функции в этой точке $f(x) = 6.544854505667396e-05$.