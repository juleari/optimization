\section{Методы последовательной безусловной оптимизации. Метод штрафов}
    \subsection{Постановка задачи}
        Даны дважды непрерывно дифференцируемые целевая функция $f(x)=f(x_1 \ldots x_n)$ и функции ограничений $g_j(x) = 0, j=1 \ldots m$ и $g_j(x) \leq 0, j=m+1 \ldots p$, определяющие множество допустимых решений $X$.

        Требуется найти локальный минимум целевой функции на множестве $X$, т.е. такую точку $x^e \in X$, что

        \begin{gather}
            f(x^e) = \min_{x \in X}f(x) \nonumber
        \end{gather}
        \begin{numcases}{}
            g_j(x) = 0,    j = 1 \ldots m; m < n \nonumber \\
            g_j(x) \leq 0, j = m + 1 \ldots p \nonumber
        \end{numcases}

        Где $f(x), g(x)$:
        \begin{gather}
            f(x) = x_1^2 + x_2^2 \rightarrow min \nonumber \\
            g_1(x) = -x_1 + 1 \leq 0 \nonumber \\
            g_2(x) = x_1 + x_2 - 2 \leq 0 \nonumber 
        \end{gather}

    \subsection{Код программы}
        \lstinputlisting[language=Python]{3.1/lab.py}

    \subsection{Результат работы программы}
        Для поиска безусловного минимума используется программа, написанная для лабораторной работы 2.2.

        В результате работы программы получена точка [1.0, 0.0].
        Значение функции в этой точке: 1.0.