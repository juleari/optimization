\documentclass[12pt]{article}

\usepackage{ucs}
\usepackage{amsmath}                    % gather* для вывода формул по центру страницы
\usepackage[utf8x]{inputenc} 		% Включаем поддержку UTF8
\usepackage[russian]{babel}  		% Включаем пакет для поддержки русского языка

\usepackage[
	left=2cm, 			% Поле левое : 200 мм
	right=2cm, 			% Поле правое : 200 мм
	top=2cm,			% Поле верхнее: 200 мм
	bottom=2cm,			% Поле нижнее : 200 мм
	bindingoffset=0cm]{geometry}

\setlength{\parindent}{1.25cm}          % Абзацный отступ: 1,25 см
\usepackage{indentfirst}                % 1-й абзац имеет отступ

\usepackage[pdftex]{graphicx, color}
\usepackage{color}
\usepackage{tikz}
\usepackage{url}			% использование URL в библиографии
\usepackage{listings}			% использование листингов кода
\usepackage[nooneline]{caption}
\captionsetup[table]{justification=raggedleft}
\captionsetup[figure]{justification=centering,labelsep=endash}
%\usepackage{array}

\usepackage[nodisplayskipstretch]{setspace}
\setstretch{1.5}

\usepackage{caption}
\usepackage{graphicx}
\usepackage{subcaption}
\usepackage{cases}

\renewcommand{\baselinestretch}{1.5}

% вставка листингов с кодом
\lstset{inputencoding=utf8x,
		extendedchars=false,
		keepspaces=true,
		language=c}

\renewcommand{\lstlistingname}{Листинг}

\setcounter{tocdepth}{4} 	% chapter, section, subsection, subsubsection и paragraph
\setcounter{secnumdepth}{4}

\parindent=1,25cm				% красная строка = 1 см
\usepackage{enumitem}
\setlist[enumerate,1]{leftmargin=2.25cm}
\setlist[itemize]{leftmargin=2.25cm}
\graphicspath{{pics/}}
\DeclareGraphicsExtensions{{.jpg}}


\begin{document}

        \begin{titlepage}
		\title{Методы оптимизации}
		\date{29 Октября, 2015}
		\author{Волкова Юлия}
		\maketitle
	    \end{titlepage}

        \section{Постановка задачи}
    Найти экстремум функции
    \begin{gather}
    \label{gather:main}
    f(x) = -x_1^2 - x_2^2 - x_3^2 + x_{1}x_2 + 2x_3  
    \end{gather}
    
    Необходимо на множестве $R^n$ найти:
    \begin{enumerate}
    \item Записать необходимые условия экстремума первого порядка.
    \item Проверить выполнение достаточных и необходимых условий второго порядка в каждой стационарной точке двумя способами.
    \item Найти все стационарные точки и значения функций соответствующие этим точкам. 
    \end{enumerate}
    \newpage
\section{Решение}
    \subsection{Поиск экстремума первого порядка}
    Необходимое условие экстремума 1-ого порядка в точке $x^e$ -- равенство нулю градиента ф-ции $f$ в этой точке. Вычислим градиент ф-ции $f$:
    \begin{gather}
    \nabla f(x)= <-2x_1 - 1 + x_2; -2x_2 + x_1; -2x_3 + 2> \nonumber \\
    \nabla f(x) = 0 \nonumber
    \end{gather}
    \begin{numcases}{}
        -2x_1 - 1 + x_2 = 0 \nonumber \\
        -2x_2 + x_1 = 0 \nonumber \\
        -2x_3 + 2 = 0 \nonumber
    \end{numcases}
    
    Из последнего уравнения можно сделать вывод, что $x_3 = 1$, из предпоследнего - что $x_1 = 2x_2$. Тогда
    \begin{gather}
        -4x_2 - 1 + x_2 = 0 \nonumber \\
        -3x_2 = 1 \nonumber \\
        x_2 = -1/3 \nonumber \\
        x_1 = -2/3 \nonumber
    \end{gather}
    
    Из этого множества решений, решением системы является $x^e = (x_1, x_2, x_3) = (-1/3,-2/3,1)$ --- единственная стационарная точка уравнения \ref{gather:main}.
        \section{Решение}
Обобщённая функция Лагранжа:
\begin{gather}
    L(x, \lambda_0, \lambda_1) = \lambda_0(x_1^2 - 4x_1 + 4 + x_2^2 - 6x_2 + 9 + x_1x_2)\nonumber
\end{gather}

Необходимые условия экстремума:
\begin{gather}
    \frac{\delta L(x, \lambda_0, \lambda_1)}{\delta x_i} = 0, i = 1 \ldots n \nonumber \\
    \frac{\delta L(x, \lambda_0, \lambda_1)}{\delta x_1} = \lambda_0(2x_1 - 4 + x_2) + \lambda_1(2x_1) = 0 \nonumber \\
    \frac{\delta L(x, \lambda_0, \lambda_1)}{\delta x_2} = \lambda_0(2x_1 - 4 + x_2) + \lambda_1(2x_1) = 0 \nonumber \\
    \lambda_1(x_1^2 + x_2^2 - 52) = 0 \nonumber
\end{gather}

Пусть $\lambda_0 = 0$:
\begin{gather}
    2\lambda_1 x_1 = 0 \nonumber \\
    2\lambda_1 x_2 = 0 \nonumber \\
    x_1 = x_2 = 0 \nonumber
\end{gather}

Пусть $\lambda_1 = 1$:
\begin{gather}
    \lambda_1 = -\frac{\lambda_0(2x_2 - 6 + x_1)}{2x_2} \nonumber \\
    6x_1 - 4x_2 - x_1^2 + x_2^2 = 0 \nonumber \\
    x_2 = 2 \pm \sqrt{x_1^2 - 6x_1 + 4} \nonumber \\
    x_1^2 - 3x_1 + 2 = 2(1 - x)\sqrt{x_1^2 - 6x_1 + 4}\nonumber \\
    x_1 = 1, x_1 = \frac23, x_1 = 6 \nonumber \\
    x_2 = 2\frac23 \nonumber
\end{gather} 

Получили две точки $x = [0, 0]$, $x = [\frac23, 2\frac23]$.

Проверим достаточные условия первого порядка: 
число ограничений-неравенств и число активных ограничений-неравенств совпадают и равны~$1$.
$\lambda_1 \geq 0$ значит -- локальные минимумы.

Проверим достаточные условия второго порядка:
\begin{gather}
    d^2L(x, \lambda_0, \lambda_1) = \lambda_0 + 2\lambda_0 + \lambda_1 + 2\lambda_0 + \lambda_1 + \lambda_0 \geq 0 \nonumber
\end{gather} 

$d^2L(x, \lambda_0, \lambda_1) \geq 0$ -- значит локальные минимумы.

Высчитываем значения в точках $x = [0, 0]$, $x = [\frac23, 2\frac23]$:
\begin{gather}
    f(0, 0) = 4 + 9 + 0 = 13 \nonumber \\
    f(\frac23, 2\frac23) = \frac{16}9 + \frac19 + \frac23\frac83=\frac{11}3 \nonumber
\end{gather} 
        \section{Поиск значений ф-ции в стационарных точках}
    Вычислим значение ф-ции \ref{gather:main} в стационарной точке $x^e=(-2/3,-1/3,1)$, и получим
    \begin{gather}
        f(x^e) = -4/9 - 1/9 - 1 + 2/3 + 2/9 + 2 = 4/3
    \end{gather}
    
    
        
\end{document}