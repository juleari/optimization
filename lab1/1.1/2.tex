\subsection{Проверка необходимых и достаточных условий в стационарных точках}
    Проверим необходимое условие экстремума первого порядка в стационарных точках. 
    \subsubsection{Первый способ (с помощью угловых миноров)}
        Необходимо убедиться, что матрица Гессе в точке $x^e$  является положительно (отрицательно) полуопределенной, т.е.
        \begin{gather}
        H(x^e) \geq 0 \nonumber \\
        H(x^e) \leq 0 \nonumber
        \end{gather}
        
        Вычислим матрицу Гессе от уравнения \ref{gather:main}:
        \[
        H(f(x)) = 
        \begin{pmatrix}
            -2 &  1 & 0 \\
             1 & -2 & 0 \\
             0 &  0 & -2
        \end{pmatrix}\]
        
        При подстановке $x^e$, матрица Гессе для будет иметь вид:
        \[
        H(f(x^e)) =
        \begin{pmatrix}
            -2 &  1 & 0 \\
             1 & -2 & 0 \\
             0 &  0 & -2
        \end{pmatrix}\]
        
        Пользуясь критерием Сильвестра, вычислим угловые минторы матрицы Гессе, и получим
        \[
        \Delta_1 = -2 < 0
        \]
        
        \[
        \Delta_2 = 
        \begin{vmatrix}
            -2 &  1 \\
             1 & -2 
        \end{vmatrix} = 4 - 1 = 3 > 0
        \]
        
        \[
        \Delta_3 = 
        \begin{vmatrix}
            -2 &  1 & 0 \\
             1 & -2 & 0 \\
             0 &  0 & -2
        \end{vmatrix} = -8 + 2 = -6 < 0
        \]
        
        Согласно вычислениям $\Delta_i, i=\overline{1..3}$, матрица Гессе в точке $x^e$ является отрицательно определенной.
    
    \subsubsection{Второй способ (с помощью собственных значений матрицы Гессе)}
        Собственные значения матрицы Гессе $\lambda_i, i = \overline{1..n}$, должны быть положительно определенными в в стационарных точках
        т.е. $\forall \lambda_i > 0, i = \overline{1..n} $
        \[
        |H(x^e) - \lambda E| =  
        \begin{vmatrix}
            -2 - \lambda & 1 & 0 \\
            1 & -2 - \lambda & 0 \\
            0 & 0 & -2 - \lambda
        \end{vmatrix} = 0
        \]
        
        Получаем уравнение
        \begin{gather}
            -(\lambda^3 + 6\lambda^2 + 11\lambda + 6) = 0 \nonumber
        \end{gather}
        Вещественными корнями этого уравнения являются $\lambda_1 = -1$, $\lambda_2 = -2$, $\lambda_3 = -3$. Таким образом, в $R^N$, все решения данного уравнения являются отрицательными.