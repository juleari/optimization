\section{Постановка задачи}
    Найти экстремум функции
    \begin{gather}
    \label{gather:main}
    f(x) = -x_1^2 - x_2^2 - x_3^2 + x_{1}x_2 + 2x_3  
    \end{gather}
    
    Необходимо на множестве $R^n$ найти:
    \begin{enumerate}
    \item Записать необходимые условия экстремума первого порядка.
    \item Проверить выполнение достаточных и необходимых условий второго порядка в каждой стационарной точке двумя способами.
    \item Найти все стационарные точки и значения функций соответствующие этим точкам. 
    \end{enumerate}
    \newpage
\section{Решение}
    \subsection{Поиск экстремума первого порядка}
    Необходимое условие экстремума 1-ого порядка в точке $x^e$ -- равенство нулю градиента ф-ции $f$ в этой точке. Вычислим градиент ф-ции $f$:
    \begin{gather}
    \nabla f(x)= <-2x_1 - 1 + x_2; -2x_2 + x_1; -2x_3 + 2> \nonumber \\
    \nabla f(x) = 0 \nonumber
    \end{gather}
    \begin{numcases}{}
        -2x_1 - 1 + x_2 = 0 \nonumber \\
        -2x_2 + x_1 = 0 \nonumber \\
        -2x_3 + 2 = 0 \nonumber
    \end{numcases}
    
    Из последнего уравнения можно сделать вывод, что $x_3 = 1$, из предпоследнего - что $x_1 = 2x_2$. Тогда
    \begin{gather}
        -4x_2 - 1 + x_2 = 0 \nonumber \\
        -3x_2 = 1 \nonumber \\
        x_2 = -1/3 \nonumber \\
        x_1 = -2/3 \nonumber
    \end{gather}
    
    Из этого множества решений, решением системы является $x^e = (x_1, x_2, x_3) = (-1/3,-2/3,1)$ --- единственная стационарная точка уравнения \ref{gather:main}.